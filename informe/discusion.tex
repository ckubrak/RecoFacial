\par Las expectativas que teníamos respecto de las pruebas usando solo KNN en comparación a KNN + PCA era que la segunda iba a dar mejores resultados en cuanto a las métricas de reconocimiento, sin embargo los resultados resultaron bastante similares.

\par Lo que sí se logra usando PCA es comprimir la base de datos de imágenes durante el preprocesamiento. De esta manera en la fase de reconocimiento se trabaja con matrices más chicas, lo cual es útil si se trabaja con una base de datos con imágenes grandes o con muchas imágenes.

\par Con respecto a los tiempos tal como esperábamos PCA resulta lento en el procesamiento de la base de datos de entrenamiento sobre todo cuando se agregan muchas componentes principales, pero siendo que esto únicamente es necesario realizarlo cuando modifican las imágenes de la base de datos, no influye en la etapa de reconocimiento. Teniendo esto en cuenta consideramos que PCA es aplicable cuando se trata de una base de datos que permanece relativamente estática a lo largo del tiempo. Lo vemos viable para aplicar en una empresa ya que suponemos que no se estarían agregando eliminando empleados de forma tan frecuente.

\par También suponíamos que las primeras componentes principales iban a influir más en tener buenos resultados de reconocimiento. Esto efectivamente fue así y lo usamos al diseñar los casos de test usando $\alpha$ más próximos en los valores pequeños y espaciándolos en valores más altos.
\par Una de las cosas que suponíamos es que usando un K más alto en KNN iba a funcionar mejor, sin embargo obtuvimos mejores métricas para K más chicos.
