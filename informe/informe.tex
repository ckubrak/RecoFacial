%\documentclass[a4paper]{article}
\documentclass[10pt,a4paper]{article}
\usepackage[utf8]{inputenc} % para poder usar tildes en archivos UTF-8
\usepackage[spanish,es-tabla]{babel}
\usepackage{verbatim}
\usepackage{clrscode3e}
\usepackage{amssymb}
\usepackage{graphicx}
\usepackage{float}
\usepackage{pdfpages}
\usepackage{subcaption} %  for subfigures environments 


% \usepackage{bibtex}

%\usepackage{a4wide} % márgenes un poco más anchos que lo usual

\usepackage{caratula} % Se puede descargar en ~> https://github.com/bcardiff/dc-tex
\usepackage[breaklinks=true]{hyperref}


\begin{document} % Todo lo que escribamos a partir de aca va a aparecer en el documento.

%fran
%\sloppy

% Completar los datos de la caratula
\titulo{Trabajo Práctico 2 - Reconocimiento de Imagenes} 
\fecha{\today}
\materia{Métodos Numéricos}
\grupo{Grupo "Greco - Herrera - Kubrak"}

% Completar los integrantes del grupo:)
%\integrante{Facundo, Araujo}{321/15}{facalj\_velez@hotmail.com}
\integrante{Cristian, Kubrak}{456/15}{Kubrakcristian@gmail.com}
\integrante{Marcela Alejandra, Herrera}{1162/84}{marcelaalejandraherrera@yahoo.com.ar}
\integrante{Luis Fernando, Greco}{150/15}{luifergreco@gmail.com}


\maketitle
\par \textbf{Abstract:} El objetivo del Trabajo es realizar reconocimiento de imágenes de rostros mediante aprendizaje supervisado. 
\par Se utiliza K-fold cross validation para generar las muestras de prueba y de validación a partir de la base de datos disponible. Durante la fase de reconocimiento se utiliza KNN para determinar a qué individuo pertenece la imagen. Como técnica auxiliar para preprocesar las imágenes se aplica PCA como forma de optimizar el uso del espacio de almacenamiento. Para la obtención de las componentes principales en PCA se utiliza el algoritmo de Deflación.
\par Para la evaluación de los resultados se utilizan las métricas de accuracy y recall.

\par  \textbf{Palabras clave:} Reconocimiento de rostros - Aprendizaje supervisado - PCA - KNN - KFold - Método de la potencia - Algoritmo de deflación - Accuracy - Recall.


% Aca comienzan a escribir su informe
\tableofcontents

\newpage

\section{Introducción}
\subsubsection*{Introducción}
\par El problema que se nos platea es el de realizar un reconocimiento facial. Partimos de una base de datos, a la cual llamaremos \textit{base de entrenamiento},
que consiste en un conjunto de $N$ personas de las cuales contamos con $M$ fotos diferentes de sus caras. Al recibir una nueva imagen, buscamos identificar
a qu\'e persona le correspone. 
\par Para reconocer la nueva cara experimentaremos con dos m\'etodos: el primero usando los $k$ vecinos m\'as cercanos (\textit{kNN}) y 
el segundo utilizando el an\'alisis de componentes principales (\textit{PCA}) como forma de preprocesar la \textit{base de entrenamiento} para reducir el 
tama\~{n}o de la imagen y luego correr \textit{kNN}. 
\par Por \'ultimo, para evaluar los m\'etodos y la correcta elecci\'on de par\'ametros utilizaremos una t\'ecnica de \textit{cross validation} llamado
\textit{K-fold}.

\subsubsection*{K vecinos m\'as cercanos}
A partir de la \textit{base de entrenamiento} buscamos identificar a qu\'e sujeto pertenece una nueva cara sin identificar.
Para este algoritmo considera a cada imagen de la \textit{base de entrenamiento} como un vector de dimensi\'on $n$, donde $n = altura*ancho$ asumiendo que
todas las im\'agenes tienen el mismo tama\~no, para el cual se conoce a cu\'al es el ID de la imagen (i.e.: la persona) para luego mediante el c\'alculo
de la nomrma de la diferencia, obtener los $k$ elementos m\'as cercanos.\\
Esta forma de encarar el problema resuta poco pr\'actica cuando la dimensi\'on de la im\'agen es grande, es por esto que en ciertos casos preprocesamos
la base de datos con el m\'etodo \textit{PCA}.


\subsubsection*{An\'alisis de componentes principales}
Para este algoritmo de preprocesamiento, calculamos $\mu = (\sum_{i=1}^{n})/n$ el promedio de todas la im\'agenes
\newpage

% \section{Demostraciones}
% \newpage

\section{Desarrollo}
%\subsubsection*{Desarrollo}

\textbf{Generalidades}

Para experimentar analizamos la influencia en los resultados de las diferentes variables de experimentación que manejamos utilizando tanto el método KNN como el KNN + PCA. 
Las mismas son la cantidad de vecinos considerados por KNN (k), y la cantidad de componentes principales de la imagen transformada ($\alpha$).
Lo hicimos sobre dos bases de datos: una con im\'agenes de tamaño reducido, y otra con imágenes sin reducir.
Para manejar las imágenes implementamos la clase Imagen, en la cual almacenamos la imagen y demás parámetros necesarios de cada una de ellas.
Para la lectura de los archivos de imágenes utilizamos las librerías provistas por la cátedra.
Cada imagen leída se vectoriza, y con el conjunto de imágenes vectorizadas se arma una estructura matricial, que resultó práctica para la implementación de los algoritmos requeridos.

\textbf{KNN}

El KNN es de algúna forma el centro del desarrollo, ya que es donde se toma la decisión de cuál es el sujeto de la base de entrenamiento que se corresponde con la imagen que elegimos para testear.

Esto lo hicimos calculando la distancia euclídea entre el elemento a testear y todos los elementos de la base de entrenamiento, guardando la distancia resultante y el sujeto al que pertenecía cada imagen de la base. Luego ordenamos los datos de acuerdo a las distancias obtenidas. Con los datos ya ordenados tomamos los id correspondientes a las k menores distancias y finalmente calculamos la moda, es decir, el ID que  aparecía más veces. En otras palabras quién es el individuo que "matchea" mejor con la imagen que testeamos según nuestros métodos. 

\par Una de las dificultades con que nos encontramos fue el decidir los tipos de datos a utilizar. Las imágenes vienen codificadas con números enteros entre 0 y 255 y para trabajar con KNN (sin PCA) utilizamos enteros que ocupan menos espacio de memoria. Sin embargo por la naturaleza de los cálculos necesarios para PCA tuvimos que utilizar doubles. Esto nos llevó a tener que duplicar parte del código para adaptarse al nuevo tipo de dato.


\textbf{PCA}

La implementación del método PCA consta de varios pasos:
En un primer paso se arma la matriz X (de tipo matriz de doubles) que contiene los bytes de datos de las imágenes vectorizadas. Esta matriz se debería usar para calcular la matriz de covarianza de la muestra, que luego debería ser diagonalizada para obtener los autovectores que permiten hacer una cambio de base de la muestra dejando las variables (valores de las columnas) lo más independientes posible.
La matriz de covarianza se calcula haciendo el producto matricial $X^{t} * X$. Considerando que la cantidad de columnas de la matriz $X$ es igual a la cantidad de pixeles de una imagen, el resultado del producto es una matriz que fácilmente puede volverse gigantesca como para poder operar con ella, de manera que resulta muy costoso en términos de uso de la memoria del equipo. Con el objeto de atenuar este problema, en lugar de calcular los autovectores de la matriz $X^{t} * X$, lo hacemos con $X * X^{t}$.
\par Estas matrices comparten los mismos autovalores y sus autovectores puede ser calculados unos en función de los otros, tal como se detalla en la Introducción teórica. Si bien la implementación se complicó porque tuvimos que programar la conversión de los autovectores, la ventaja que obtuvimos es que operamos con matrices de menor dimensión, puesto que la cantidad de imágenes (filas de X) es en nuestro caso notablemente menor que la cantidad de pixeles de cada imagen (columnas de X).
La matriz $X * X^{t}$ es una matriz simétrica, en un principio nos planteamos la posibilidad de usar alguna estrategia de almacenamiento que aprovechara esta condición para ahorrar memoria, pero en este caso decidimos no hacerlo y tratarla como una matriz común para simplificar la implementación, sin embargo en caso de una implementación donde sea crítico el uso del espacio debería considerarse seriamente esta opción.

\textbf{Método de la potencia:} El método de la Potencia necesita para la iteración de un vector $x$ que inicialmente contiene un valor arbitrario, el cual decidimos generar de forma aleatoria. Durante el testeo de las funciones observamos que corridas sucesivas efectuadas con las mismas imágenes de entrada producían diferentes resultados en los autovectores generados. Las diferencias se observan en los decimales menos significativos. La hipótesis es que estas diferencias se producen por utilizar el vector inicial generado de manera aleatoria. Si bien esta forma de generarlo sirvió muy bien a la hora de calcular los autovalores y autovectores de la matriz de covarianza, no resulta bueno a la hora de reproducir los experimentos, algo de lo que nos dimos cuenta luego de la experimentación.
\par En el método de Deflación pudimos observar que al buscar autovalores y autovectores, las componentes principales tienen una convergencia más veloz, arribándose a un resultado preciso en pocas iteraciones del método de Potencia. A medida que se calculan mayor cantidad de componentes, la convergencia del método de Potencia se hace notoriamente más lenta y por momentos parecía que no había convergencia cuando en realidad sí la había. Esto nos llevó a utilizar dos criterios de parada para el cálculo de cada autovalor: en una primera instancia usamos una cantidad de iteraciones tope de 10.000, pidiendo una variación entre los resultados de dos iteraciones sucesivas menor a $10^{-5}$; si al cabo de 10.000 iteraciones no llegamos a una convergencia, continuamos iterando hasta un tope de 100.000 iteraciones más y aceptando una tolerancia menos precisa de $10^{-3}$.


\textbf{Kfold (Cross-validation)}
Decidimos utilizar 5-fold.
La determinación del valor de $k$ la realizamos teniendo en cuenta los siguientes factores:
\begin{itemize}
\item Al realizar la partición de la base de datos en datos de entrenamiento y datos de test queríamos que cada clase estuviera representada en una proporción similar, o sea que la selección fuera balanceada. Si bien al principio habíamos pensado hacer la partición de forma aleatoria, esto hubiera podido llevar a situaciones como por ejemplo tener en la base de entrenamiento todos los elementos de una clase y muy pocos de otra con lo cual el entrenamiento sería sesgado favoreciendo a la clase con más representantes. Por otro lado a la hora de testear podría no haber elementos de esa clase que no hubieran sido usados para entrenar con lo cual se invalidarían los resultados. Contrariamente los parámetros elegidos podrían no ser adecuados para la clase con menos representantes, con lo cual el reconocimientos de elementos pertenecientes a esa clase sería deficiente.

\item Teniendo en cuenta el punto anterior los posibles $k$ elegibles se reducen a los divisores de la cantidad de imágenes de cada clase. Siendo en el nuestro caso de 10 imágenes por clase los posibles valores son: 1, 2, 5 y 10. El 1 no tiene sentido en la práctica ya que equivale a no tener datos de entrenamiento. El 10 supone que queda solo un representante de cada clase para testear, entonces si la imagen que usamos para testear posee algúna anormalidad respecto de las otras de su clase afectaría negativamente nuestros resultados, es decir, nuestro resultado sería muy sensible a outsiders y poco robusto. En el caso de 2, estaríamos usando la mitad de los datos para entrenar y la mitad de los datos para testear y tratándose de una base de datos relativamente chica (con pocos representantes de cada clase) consideramos que tendríamos poca variedad de datos de entrenamiento. En el caso de $k$=5, estaríamos considerando dos imágenes de cada clase para testear y ocho para entrenar. De todas las posibilidades, nos pareció la más adecuada porque nos da una cantidad relativamente grande de datos de entrenamiento y más de una imagen para testear. 
\end{itemize}

\textbf{Métricas}

\par Las métricas que elegimos para evaluar los resultados fueron \textit{accuracy} y \textit{recall}, con el \textit{accuracy} tenemos una medida de clasificaciones correctas en general (o sea las veces que se acierta en clasificar en la clase correcta y en no clasificar en una clase que no corresponde), mientras que el \textit{recall} nos permite maximizar la cantidad de identificaciones acertadas de cada clase en relación al total de elementos de la clase presentes en la muestra.
Para calcularlas, por cada fold, para cada imagen a reconocer se guarda el ID de la clase a la que pertenece y la clasificación que hace el sistema.
\par Una vez obtenida esta información para todas las imágenes a reconocer de la totalidad de los folds, se procede a calcular los verdaderos positivos, verdaderos negativos, falsos positivos y falsos negativos. Y con estos se calculan las métricas mencionadas (como se detalló en la introducción teórica).

%\subsection{Experimentación}

\textbf{Experimentación}
Uno de los parámetros determinantes en la experimentación es la cantidad de vecinos cercanos, otro punto relevante es el valor de $\alpha$ que vamos a utilizar. Es por eso que vamos a plantear una serie de tests con estos parámetros para determinar los valores que nos ofrezcan un mejor \textit{trade off} entre las diferentes métricas que utilizaremos para evaluar nuestra implementación.\newline
Estas son Accuracy, Recall y el tiempo de ejecución.

A su vez, vamos a testear lo mencionado anteriormente tanto con la implementación de KNN como con KNN+PCA para evaluar también el funcionamiento de PCA. 

Nota: En los tests dónde evaluamos $\alpha$ tomamos K = 1 y de la misma forma, en los que evaluamos K tomamos $\alpha$ = 10. 

\par Nuestras expectativas previas a la experimentación son las siguientes\\
Por un lado consideramos que el tamaño de las im\'agenes influiría acrecentando el tiempo requerido para su procesamiento pero debido a la mayor informaci\'on disponible, con las imágenes más grandes 
funcionarían mejor los algoritmos.\\
Con respecto a la variaci\'on del $k$ en $KNN$ suponemos que los mejores resultados de reconocimiento los tendríamos con un $k$ no demasiado grande.\\
Frente a los diferentes algoritmos (KNN o PCA + KNN) creemos que la segunda opci\'on realizar\'a un mejor trabajo pero a su vez requiere de un mayor tiempo para el preprocesamiento de las imágenes, aunque el tiempo de reconocimiento suponemos que debería ser similar o menor que en knn solo.
Por \'ultimo, respecto al par\'ametro $\alpha$ (cantidad de iteraciones del m\'etodo de las potencias) resulta obvio estimar que a mayor $\alpha$, mayor ser\'a el tiempo de ejecución aunque como contrapartida esperamos que un mayor $\alpha$ mejore significativamente la tasa de reconocimiento.


\newpage

%\section{Experimentación}
%\subsubsection*{Experimentación}
Para experimentar analizamos la influencia en los resultados de las diferentes variables de experimentación que manejamos utilizando tanto el método KNN como el KNN + PCA. 
Las mismas son la cantidad de vecinos considerados por KNN (k), y la cantidad de componentes principales de la imagen transformada ($\alpha$).
Lo hicimos sobre dos bases de datos: una con im\'agenes de tamaño reducido, y otra con imágenes sin reducir (a las cuales llamaremos \textit{big tempo})





% Conclusión:
% Luego de observar estos gráficos llegamos a algunas conclusiones.
% En cuanto al tiempo, por un lado, la cantidad de vecinos cercanos que tomemos no afecta significativamente el tiempo, pero lo que sí lo afecta es el $\alpha$ de PCA.
% A qué se debe esto? Teniendo en cuenta el funcionamiento de nuestro algoritmo, entendemos que esto se debe a que una gran parte del tiempo de procesamiento de PCA se consume en el método de la potencia (que se realiza $\alpha$ veces) y en la transformación de los autovectores calculados en los de la verdadera matriz de covarianza de la muestra, que involucran numerosos cálculos matriciales.

% Sin embargo, pensamos que en una implementación real estaríamos trabajando con una única training base, y las transformaciones que llevamos a cabo en el PCA las haríamos una única vez, con lo que este costo de tiempo se pagaría solamente una vez o cuando sea necesario agregar o quitar alguna imagen, para luego realizar únicamente el reconocimiento. Usando PCA + KNN tenemos la ventaja de trabajar con imágenes de menor tamaño con el consiguiente ahorro de espacio.
%\newpage

\section{Resultados}
\subsubsection*{Resultados obtenidos}
Nota: Al ser los resultados de los experimentos sobre ambos tests muy similares, decidimos analizarlos en conjunto.



%%%%%%%%%%%%%%%%%%%%%%%%%%%%%%%%%%%%%%%%%%%%%%%%%%%%%%%%%%%%%%%%%%%%%
\begin{figure}[H]
	\centering	\includegraphics[width=0.8\textwidth]{img/alfa_pca_accu.png}
	\caption{Accuracy vs $\alpha$ con PCA + KNN}
	\label{fig:Accuracy vs Alpha con KNN + PCA}
\end{figure}

\begin{figure}[H]
	\centering	\includegraphics[width=0.8\textwidth]{img/big_alfa_pca_accu.png}
	\caption{BIG Accuracy vs $\alpha$ con PCA + KNN}
	\label{fig: BIG Accuracy vs Alpha con KNN + PCA}
\end{figure}

En este caso podemos observar una relación entre las dos variables, a medida que el $\alpha$ aumenta vemos como también lo hace nuestro accuracy.
Como expresamos anteriormente, debido al funcionamiento de PCA esperábamos que a mayor $\alpha$, mejores sean nuestros resultados (todas nuestras métricas en general) y por lo tanto nuestro accuracy también.

Pero a su vez tambien vimos que un $\alpha$ muy elevado  nos elevaría el tiempo de ejecución y a su vez en este gráfico vemos como las diferencias entre accuracy son cada vez menores (por ejemplo entre $\alpha$ = 10 y $\alpha$ = 50).

En base a los resultados obtenidos concluimos que un valor de $\alpha$ cercano a 10 nos daría un buen balance entre relativamente la cantidad de componentes principales y la efectividad (se sacrifica algo de efectividad pero a cambio trabajamos con imágenes mucho más chicas, reduciendo el tiempo de ejecución).

%%%%%%%%%%%%%%%%%%%%%%%%%%%%%%%%%%%%%%%%%%%%%%%%%%%%%%%%%%%%%%%%%%%%%

\begin{figure}[H]
	\centering	\includegraphics[width=0.8\textwidth]{img/alfa_pca_tiempo.png}
	\caption{Tiempo vs $\alpha$ con PCA + KNN}
	\label{fig:Tiempo vs Alpha con PCA + KNN}
\end{figure}

\begin{figure}[H]
	\centering	\includegraphics[width=0.8\textwidth]{img/big_alfa_pca_tiempo.png}
	\caption{Big Tiempo vs $\alpha$ con PCA + KNN}
	\label{fig:Big Tiempo vs Alpha con PCA + KNN}
\end{figure}

Tal como esperabamos vemos que a medida que el $\alpha$ aumenta (es decir, cuantas más componentes principales tengamos), el tiempo de ejecución también lo hace.

Luego, en línea con los resultados de los gráficos anteriores (Accuracy vs $\alpha$) podemos volver a afirmar que un $\alpha$ cercano a 10 sería un buen balance. En este gráfico notamos si tomaramos $\alpha$ = 30, tardaría aproximadamente el triple y no obtendríamos una mejora sustancial en el accuracy.
%%%%%%%%%%%%%%%%%%%%%%%%%%%%%%%%%%%%%%%%%%%%%%%%%%%%%%%%%%%%%%%%%%%%%


\begin{figure}[H]
	\centering	\includegraphics[width=0.8\textwidth]{img/k_knn_accu.png}
	\caption{Accuracy vs K con KNN}
	\label{fig:Accuracy vs K con KNN}
\end{figure}

\begin{figure}[H]
	\centering	\includegraphics[width=0.8\textwidth]{img/big_k_knn_accu.png}
	\caption{Big Accuracy vs K con KNN}
	\label{fig:Big Accuracy vs K con KNN}
\end{figure}

Dados los resultados, en este caso consideramos que utilizando un valor de K cercano a 10 obtenemos la mejor relación (dentro de nuestro set de tests).\newline
Por un lado evitamos el problema que ocurre cuando K es demasiado grande y por otro, tomamos una cantidad de imágenes cercanas suficiente como para minimizar el impacto de algún outsider. Aun que cabe destacar que en este caso particular K = 1 tuvo un mejor comportamiento de lo que esperabamos, consideramos que sería arriesgado tomarlo como valor confiable con otros sets de imágenes.

%%%%%%%%%%%%%%%%%%%%%%%%%%%%%%%%%%%%%%%%%%%%%%%%%%%%%%%%%%%%%%%%%%%%%

\begin{figure}[H]
	\centering	\includegraphics[width=0.8\textwidth]{img/k_knn_tiempo.png}
	\caption{Tiempo vs K con KNN}
	\label{fig:K vs Tiempo con KNN}
\end{figure}
\begin{figure}[H]
	\centering	\includegraphics[width=0.8\textwidth]{img/big_k_knn_tiempo.png}
	\caption{Big Tiempo vs K con KNN}
	\label{fig:Big K vs Tiempo con KNN}
\end{figure}

En estos tests obtuvimos resultados coherentes con lo esperado.
Lo único que podemos señalar es un tiempo mayor en PCA a lo que uno podría intuir, pero esto se debe a la aplicación de PCA en cada ejecución. En un uso real esperaríamos encontrar una reducción del tiempo de ejecución de KNN+PCA en relación al KNN.

\begin{figure}[H]
	\centering	\includegraphics[width=0.8\textwidth]{img/k_pca_tiempo.png}
	\caption{K vs Tiempo con KNN + PCA}
	\label{fig:K vs Tiempo con KNN + PCA}
\end{figure}
\begin{figure}[H]
	\centering	\includegraphics[width=0.8\textwidth]{img/big_k_pca_tiempo.png}
	\caption{Big Tiempo vs K con KNN + PCA}
	\label{fig:Big K vs Tiempo con KNN + PCA}
\end{figure}


%%%%%%%%%%%%%%%%%%%%%%%%%%%%%%%%%%%%%%%%%%%%%%%%%%%%%%%%%%%%%%%%%%%%%
\begin{figure}[H]
	\centering
	\includegraphics[width=0.8\textwidth]{img/k_pca_accu.png}
	\caption{Accuracy vs K con KNN + PCA}
	\label{fig:K vs Accuracy con KNN + PCA}
\end{figure}

En este caso vemos una estrecha relación entre cuantos vecinos cercanos tomamos y el accuracy.
Esto se debe a que al tomar más vecinos cercanos nos exponemos a un error mayor debido a que le estaríamos dando el mismo peso a todos esos K vecinos sin importar que tan cerca estén de la imagen testeada.
Llevando esto a un extremo podríamos tomar $K$ = “Tamano de matriz de entrenamiento”  cualquiera de las clases tendría el mismo peso con lo que perdería sentido este método.
Por otro lado tampoco es conveniente tener un $K$ demasiado chico. Por ejemplo, si tomaramos $K = 1$ asociaríamos la imagen a testear con la que esté a menor distancia, que debido a alguna diferencia la forma en que fue tomada la imagen puede no pertenecer a la clase de la imagen testeada.

Es llegamos a la conclusión de que utilizando un valor de K cercano a 10 obtenemos la mejor relación (dentro de nuestro set de tests).
Por un lado evitamos el problema que ocurre cuando K es demasiado grande y por otro, tomamos una cantidad de imágenes cercanas suficiente como para minimizar el impacto de algún outsider.


%%%%%%%%%%%%%%%%%%%%%%%%%%%%%%%%%%%%%%%%%%%%%%%%%%%%%%%%%%%%%%%%%%%%%
\begin{figure}[H]
	\centering	
	\includegraphics[width=0.8\textwidth]{img/acu_pre.png}
	\caption{Accuracy y precision vs K}
	\label{fig: Accuracy y precision vs K con KNN}
\end{figure}
En este último experimento estudiamos la forma en la que la cantidad de vecinos cercanos afecta a las métricas Accuracy y Precisión. Para esto utilizamos el método KNN (sin PCA) para no involucrar más variables dentro del experimento de las necesarias.

Lo que encontramos no fue muy distinto de lo esperado. Antes ya vimos la forma en la que Accuracy variaba en función de la cantidad de vecinos cercanos (Figura 5). Pero como explicamos en cuanto al funcionamiento y elección de un K apropiado para el KNN, un K = 250 por ejemplo no es una buena elección, así que en este caso el accuracy resulta ser una métrica un tanto engañosa.

Para tener un sistema preciso -valga la redundancia- necesitamos un valos de precision relativamente alto. Entonces en función de lo que nos indica el gráfico nuevamente un valor aproximado de K = 10 nos parece una buena opción.
\newpage


\section{Discusión}
\par Las expectativas que teníamos respecto de las pruebas usando solo KNN en comparación a KNN + PCA era que la segunda iba a dar mejores resultados en cuanto a las métricas de reconocimiento, sin embargo los resultados resultaron bastante similares.

\par Lo que sí se logra usando PCA es comprimir la base de datos de imágenes durante el preprocesamiento. De esta manera en la fase de reconocimiento se trabaja con matrices más chicas, lo cual es útil si se trabaja con una base de datos con imágenes grandes o con muchas imágenes.

\par Con respecto a los tiempos tal como esperábamos PCA resulta lento en el procesamiento de la base de datos de entrenamiento sobre todo cuando se agregan muchas componentes principales, pero siendo que esto únicamente es necesario realizarlo cuando modifican las imágenes de la base de datos, no influye en la etapa de reconocimiento. Teniendo esto en cuenta consideramos que PCA es aplicable cuando se trata de una base de datos que permanece relativamente estática a lo largo del tiempo. Lo vemos viable para aplicar en una empresa ya que suponemos que no se estarían agregando eliminando empleados de forma tan frecuente.

\par También suponíamos que las primeras componentes principales iban a influir más en tener buenos resultados de reconocimiento. Esto efectivamente fue así y lo usamos al diseñar los casos de test usando $\alpha$ más próximos en los valores pequeños y espaciándolos en valores más altos.
\par Una de las cosas que suponíamos es que usando un K más alto en KNN iba a funcionar mejor, sin embargo obtuvimos mejores métricas para K más chicos.

\newpage

\section{Conclusiones}
\input{conclusiones}
\newpage

\end{document}

