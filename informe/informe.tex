\documentclass[a4paper]{article}
%\usepackage{a4wide} % márgenes un poco más anchos que lo usual


\usepackage[spanish]{babel} % Le indicamos a LaTeX que vamos a escribir en español.
\usepackage[utf8]{inputenc} % Permite utilizar tildes y eñes normalmente
\usepackage{caratula} % Se puede descargar en ~> https://github.com/bcardiff/dc-tex
\usepackage[breaklinks=true]{hyperref}


\begin{document} % Todo lo que escribamos a partir de aca va a aparecer en el documento.

%fran
%\sloppy

% Completar los datos de la caratula
\titulo{Trabajo Práctico 1 - Page Rank} 
\fecha{\today}
\materia{Métodos Numéricos}
\grupo{Grupo (número de grupo)}

% Completar los integrantes del grupo:)
\integrante{Facundo, Araujo}{321/15}{facalj_velez@hotmail.com}
\integrante{Cristian, Kubrak}{456/15}{Kubrakcristian@gmail.com}
\integrante{Marcela Alejandra, Herrera}{1162/84}{marcelaalejandraherrera@yahoo.com.ar}
\integrante{Luis Fernando, Greco}{150/15}{luifergreco@gmail.com}


\maketitle

% Aca comienzan a escribir su informe
\tableofcontents

\newpage

\section{Introducción}
\subsubsection*{Introducción}
\par El problema que se nos platea es el de realizar un reconocimiento facial. Partimos de una base de datos, a la cual llamaremos \textit{base de entrenamiento},
que consiste en un conjunto de $N$ personas de las cuales contamos con $M$ fotos diferentes de sus caras. Al recibir una nueva imagen, buscamos identificar
a qu\'e persona le correspone. 
\par Para reconocer la nueva cara experimentaremos con dos m\'etodos: el primero usando los $k$ vecinos m\'as cercanos (\textit{kNN}) y 
el segundo utilizando el an\'alisis de componentes principales (\textit{PCA}) como forma de preprocesar la \textit{base de entrenamiento} para reducir el 
tama\~{n}o de la imagen y luego correr \textit{kNN}. 
\par Por \'ultimo, para evaluar los m\'etodos y la correcta elecci\'on de par\'ametros utilizaremos una t\'ecnica de \textit{cross validation} llamado
\textit{K-fold}.

\subsubsection*{K vecinos m\'as cercanos}
A partir de la \textit{base de entrenamiento} buscamos identificar a qu\'e sujeto pertenece una nueva cara sin identificar.
Para este algoritmo considera a cada imagen de la \textit{base de entrenamiento} como un vector de dimensi\'on $n$, donde $n = altura*ancho$ asumiendo que
todas las im\'agenes tienen el mismo tama\~no, para el cual se conoce a cu\'al es el ID de la imagen (i.e.: la persona) para luego mediante el c\'alculo
de la nomrma de la diferencia, obtener los $k$ elementos m\'as cercanos.\\
Esta forma de encarar el problema resuta poco pr\'actica cuando la dimensi\'on de la im\'agen es grande, es por esto que en ciertos casos preprocesamos
la base de datos con el m\'etodo \textit{PCA}.


\subsubsection*{An\'alisis de componentes principales}
Para este algoritmo de preprocesamiento, calculamos $\mu = (\sum_{i=1}^{n})/n$ el promedio de todas la im\'agenes

\section{Resultados}
\subsubsection*{Resultados obtenidos}
Nota: Al ser los resultados de los experimentos sobre ambos tests muy similares, decidimos analizarlos en conjunto.



%%%%%%%%%%%%%%%%%%%%%%%%%%%%%%%%%%%%%%%%%%%%%%%%%%%%%%%%%%%%%%%%%%%%%
\begin{figure}[H]
	\centering	\includegraphics[width=0.8\textwidth]{img/alfa_pca_accu.png}
	\caption{Accuracy vs $\alpha$ con PCA + KNN}
	\label{fig:Accuracy vs Alpha con KNN + PCA}
\end{figure}

\begin{figure}[H]
	\centering	\includegraphics[width=0.8\textwidth]{img/big_alfa_pca_accu.png}
	\caption{BIG Accuracy vs $\alpha$ con PCA + KNN}
	\label{fig: BIG Accuracy vs Alpha con KNN + PCA}
\end{figure}

En este caso podemos observar una relación entre las dos variables, a medida que el $\alpha$ aumenta vemos como también lo hace nuestro accuracy.
Como expresamos anteriormente, debido al funcionamiento de PCA esperábamos que a mayor $\alpha$, mejores sean nuestros resultados (todas nuestras métricas en general) y por lo tanto nuestro accuracy también.

Pero a su vez tambien vimos que un $\alpha$ muy elevado  nos elevaría el tiempo de ejecución y a su vez en este gráfico vemos como las diferencias entre accuracy son cada vez menores (por ejemplo entre $\alpha$ = 10 y $\alpha$ = 50).

En base a los resultados obtenidos concluimos que un valor de $\alpha$ cercano a 10 nos daría un buen balance entre relativamente la cantidad de componentes principales y la efectividad (se sacrifica algo de efectividad pero a cambio trabajamos con imágenes mucho más chicas, reduciendo el tiempo de ejecución).

%%%%%%%%%%%%%%%%%%%%%%%%%%%%%%%%%%%%%%%%%%%%%%%%%%%%%%%%%%%%%%%%%%%%%

\begin{figure}[H]
	\centering	\includegraphics[width=0.8\textwidth]{img/alfa_pca_tiempo.png}
	\caption{Tiempo vs $\alpha$ con PCA + KNN}
	\label{fig:Tiempo vs Alpha con PCA + KNN}
\end{figure}

\begin{figure}[H]
	\centering	\includegraphics[width=0.8\textwidth]{img/big_alfa_pca_tiempo.png}
	\caption{Big Tiempo vs $\alpha$ con PCA + KNN}
	\label{fig:Big Tiempo vs Alpha con PCA + KNN}
\end{figure}

Tal como esperabamos vemos que a medida que el $\alpha$ aumenta (es decir, cuantas más componentes principales tengamos), el tiempo de ejecución también lo hace.

Luego, en línea con los resultados de los gráficos anteriores (Accuracy vs $\alpha$) podemos volver a afirmar que un $\alpha$ cercano a 10 sería un buen balance. En este gráfico notamos si tomaramos $\alpha$ = 30, tardaría aproximadamente el triple y no obtendríamos una mejora sustancial en el accuracy.
%%%%%%%%%%%%%%%%%%%%%%%%%%%%%%%%%%%%%%%%%%%%%%%%%%%%%%%%%%%%%%%%%%%%%


\begin{figure}[H]
	\centering	\includegraphics[width=0.8\textwidth]{img/k_knn_accu.png}
	\caption{Accuracy vs K con KNN}
	\label{fig:Accuracy vs K con KNN}
\end{figure}

\begin{figure}[H]
	\centering	\includegraphics[width=0.8\textwidth]{img/big_k_knn_accu.png}
	\caption{Big Accuracy vs K con KNN}
	\label{fig:Big Accuracy vs K con KNN}
\end{figure}

Dados los resultados, en este caso consideramos que utilizando un valor de K cercano a 10 obtenemos la mejor relación (dentro de nuestro set de tests).\newline
Por un lado evitamos el problema que ocurre cuando K es demasiado grande y por otro, tomamos una cantidad de imágenes cercanas suficiente como para minimizar el impacto de algún outsider. Aun que cabe destacar que en este caso particular K = 1 tuvo un mejor comportamiento de lo que esperabamos, consideramos que sería arriesgado tomarlo como valor confiable con otros sets de imágenes.

%%%%%%%%%%%%%%%%%%%%%%%%%%%%%%%%%%%%%%%%%%%%%%%%%%%%%%%%%%%%%%%%%%%%%

\begin{figure}[H]
	\centering	\includegraphics[width=0.8\textwidth]{img/k_knn_tiempo.png}
	\caption{Tiempo vs K con KNN}
	\label{fig:K vs Tiempo con KNN}
\end{figure}
\begin{figure}[H]
	\centering	\includegraphics[width=0.8\textwidth]{img/big_k_knn_tiempo.png}
	\caption{Big Tiempo vs K con KNN}
	\label{fig:Big K vs Tiempo con KNN}
\end{figure}

En estos tests obtuvimos resultados coherentes con lo esperado.
Lo único que podemos señalar es un tiempo mayor en PCA a lo que uno podría intuir, pero esto se debe a la aplicación de PCA en cada ejecución. En un uso real esperaríamos encontrar una reducción del tiempo de ejecución de KNN+PCA en relación al KNN.

\begin{figure}[H]
	\centering	\includegraphics[width=0.8\textwidth]{img/k_pca_tiempo.png}
	\caption{K vs Tiempo con KNN + PCA}
	\label{fig:K vs Tiempo con KNN + PCA}
\end{figure}
\begin{figure}[H]
	\centering	\includegraphics[width=0.8\textwidth]{img/big_k_pca_tiempo.png}
	\caption{Big Tiempo vs K con KNN + PCA}
	\label{fig:Big K vs Tiempo con KNN + PCA}
\end{figure}


%%%%%%%%%%%%%%%%%%%%%%%%%%%%%%%%%%%%%%%%%%%%%%%%%%%%%%%%%%%%%%%%%%%%%
\begin{figure}[H]
	\centering
	\includegraphics[width=0.8\textwidth]{img/k_pca_accu.png}
	\caption{Accuracy vs K con KNN + PCA}
	\label{fig:K vs Accuracy con KNN + PCA}
\end{figure}

En este caso vemos una estrecha relación entre cuantos vecinos cercanos tomamos y el accuracy.
Esto se debe a que al tomar más vecinos cercanos nos exponemos a un error mayor debido a que le estaríamos dando el mismo peso a todos esos K vecinos sin importar que tan cerca estén de la imagen testeada.
Llevando esto a un extremo podríamos tomar $K$ = “Tamano de matriz de entrenamiento”  cualquiera de las clases tendría el mismo peso con lo que perdería sentido este método.
Por otro lado tampoco es conveniente tener un $K$ demasiado chico. Por ejemplo, si tomaramos $K = 1$ asociaríamos la imagen a testear con la que esté a menor distancia, que debido a alguna diferencia la forma en que fue tomada la imagen puede no pertenecer a la clase de la imagen testeada.

Es llegamos a la conclusión de que utilizando un valor de K cercano a 10 obtenemos la mejor relación (dentro de nuestro set de tests).
Por un lado evitamos el problema que ocurre cuando K es demasiado grande y por otro, tomamos una cantidad de imágenes cercanas suficiente como para minimizar el impacto de algún outsider.


%%%%%%%%%%%%%%%%%%%%%%%%%%%%%%%%%%%%%%%%%%%%%%%%%%%%%%%%%%%%%%%%%%%%%
\begin{figure}[H]
	\centering	
	\includegraphics[width=0.8\textwidth]{img/acu_pre.png}
	\caption{Accuracy y precision vs K}
	\label{fig: Accuracy y precision vs K con KNN}
\end{figure}
En este último experimento estudiamos la forma en la que la cantidad de vecinos cercanos afecta a las métricas Accuracy y Precisión. Para esto utilizamos el método KNN (sin PCA) para no involucrar más variables dentro del experimento de las necesarias.

Lo que encontramos no fue muy distinto de lo esperado. Antes ya vimos la forma en la que Accuracy variaba en función de la cantidad de vecinos cercanos (Figura 5). Pero como explicamos en cuanto al funcionamiento y elección de un K apropiado para el KNN, un K = 250 por ejemplo no es una buena elección, así que en este caso el accuracy resulta ser una métrica un tanto engañosa.

Para tener un sistema preciso -valga la redundancia- necesitamos un valos de precision relativamente alto. Entonces en función de lo que nos indica el gráfico nuevamente un valor aproximado de K = 10 nos parece una buena opción.

\section{Discusión}
\par Las expectativas que teníamos respecto de las pruebas usando solo KNN en comparación a KNN + PCA era que la segunda iba a dar mejores resultados en cuanto a las métricas de reconocimiento, sin embargo los resultados resultaron bastante similares.

\par Lo que sí se logra usando PCA es comprimir la base de datos de imágenes durante el preprocesamiento. De esta manera en la fase de reconocimiento se trabaja con matrices más chicas, lo cual es útil si se trabaja con una base de datos con imágenes grandes o con muchas imágenes.

\par Con respecto a los tiempos tal como esperábamos PCA resulta lento en el procesamiento de la base de datos de entrenamiento sobre todo cuando se agregan muchas componentes principales, pero siendo que esto únicamente es necesario realizarlo cuando modifican las imágenes de la base de datos, no influye en la etapa de reconocimiento. Teniendo esto en cuenta consideramos que PCA es aplicable cuando se trata de una base de datos que permanece relativamente estática a lo largo del tiempo. Lo vemos viable para aplicar en una empresa ya que suponemos que no se estarían agregando eliminando empleados de forma tan frecuente.

\par También suponíamos que las primeras componentes principales iban a influir más en tener buenos resultados de reconocimiento. Esto efectivamente fue así y lo usamos al diseñar los casos de test usando $\alpha$ más próximos en los valores pequeños y espaciándolos en valores más altos.
\par Una de las cosas que suponíamos es que usando un K más alto en KNN iba a funcionar mejor, sin embargo obtuvimos mejores métricas para K más chicos.


\section{Conclusiones}
\input{conclusiones}

\section{Apendices}
\subsubsection*{Apéndices}
\textbf{Estructuras de Datos}

Para manejar las imágenes implementamos la clase Imagen, en la cual almacenamos los parámetros de cada imagen (la altura y ancho originales de la imagen, el nombre del archivo, id de la clase y los bytes de la imagen propiamente dicha en formato vector de unsigned char).
Para la lectura de los archivos de imágenes utilizamos las librerías provistas por la cátedra.
Las imágenes a medida que se leen se almacenan en la estructura que denominamos baseDeDatos, la cual consiste de un vector de Imagen.
Para la implementación de las funciones necesarias para PCA trabajamos con dos estructuras principales: doubleVector que como lo indica su nombre consiste de un vector de doubles donde almacenamos cada imagen vectorizada, y la otra estructura es dobleMatrix que usamos para almacenar todas las imágenes vectorizadas (una por fila), con el objetivo de poder operar matricialmente con ellas dentro de las funciones de una forma práctica.



\section{Referencias}
\input{referencias}

% \subsubsection*{Introducción}
\par El problema que se nos platea es el de realizar un reconocimiento facial. Partimos de una base de datos, a la cual llamaremos \textit{base de entrenamiento},
que consiste en un conjunto de $N$ personas de las cuales contamos con $M$ fotos diferentes de sus caras. Al recibir una nueva imagen, buscamos identificar
a qu\'e persona le correspone. 
\par Para reconocer la nueva cara experimentaremos con dos m\'etodos: el primero usando los $k$ vecinos m\'as cercanos (\textit{kNN}) y 
el segundo utilizando el an\'alisis de componentes principales (\textit{PCA}) como forma de preprocesar la \textit{base de entrenamiento} para reducir el 
tama\~{n}o de la imagen y luego correr \textit{kNN}. 
\par Por \'ultimo, para evaluar los m\'etodos y la correcta elecci\'on de par\'ametros utilizaremos una t\'ecnica de \textit{cross validation} llamado
\textit{K-fold}.

\subsubsection*{K vecinos m\'as cercanos}
A partir de la \textit{base de entrenamiento} buscamos identificar a qu\'e sujeto pertenece una nueva cara sin identificar.
Para este algoritmo considera a cada imagen de la \textit{base de entrenamiento} como un vector de dimensi\'on $n$, donde $n = altura*ancho$ asumiendo que
todas las im\'agenes tienen el mismo tama\~no, para el cual se conoce a cu\'al es el ID de la imagen (i.e.: la persona) para luego mediante el c\'alculo
de la nomrma de la diferencia, obtener los $k$ elementos m\'as cercanos.\\
Esta forma de encarar el problema resuta poco pr\'actica cuando la dimensi\'on de la im\'agen es grande, es por esto que en ciertos casos preprocesamos
la base de datos con el m\'etodo \textit{PCA}.


\subsubsection*{An\'alisis de componentes principales}
Para este algoritmo de preprocesamiento, calculamos $\mu = (\sum_{i=1}^{n})/n$ el promedio de todas la im\'agenes
% \input{demostraciones.tex}
% \subsubsection*{Resultados obtenidos}
Nota: Al ser los resultados de los experimentos sobre ambos tests muy similares, decidimos analizarlos en conjunto.



%%%%%%%%%%%%%%%%%%%%%%%%%%%%%%%%%%%%%%%%%%%%%%%%%%%%%%%%%%%%%%%%%%%%%
\begin{figure}[H]
	\centering	\includegraphics[width=0.8\textwidth]{img/alfa_pca_accu.png}
	\caption{Accuracy vs $\alpha$ con PCA + KNN}
	\label{fig:Accuracy vs Alpha con KNN + PCA}
\end{figure}

\begin{figure}[H]
	\centering	\includegraphics[width=0.8\textwidth]{img/big_alfa_pca_accu.png}
	\caption{BIG Accuracy vs $\alpha$ con PCA + KNN}
	\label{fig: BIG Accuracy vs Alpha con KNN + PCA}
\end{figure}

En este caso podemos observar una relación entre las dos variables, a medida que el $\alpha$ aumenta vemos como también lo hace nuestro accuracy.
Como expresamos anteriormente, debido al funcionamiento de PCA esperábamos que a mayor $\alpha$, mejores sean nuestros resultados (todas nuestras métricas en general) y por lo tanto nuestro accuracy también.

Pero a su vez tambien vimos que un $\alpha$ muy elevado  nos elevaría el tiempo de ejecución y a su vez en este gráfico vemos como las diferencias entre accuracy son cada vez menores (por ejemplo entre $\alpha$ = 10 y $\alpha$ = 50).

En base a los resultados obtenidos concluimos que un valor de $\alpha$ cercano a 10 nos daría un buen balance entre relativamente la cantidad de componentes principales y la efectividad (se sacrifica algo de efectividad pero a cambio trabajamos con imágenes mucho más chicas, reduciendo el tiempo de ejecución).

%%%%%%%%%%%%%%%%%%%%%%%%%%%%%%%%%%%%%%%%%%%%%%%%%%%%%%%%%%%%%%%%%%%%%

\begin{figure}[H]
	\centering	\includegraphics[width=0.8\textwidth]{img/alfa_pca_tiempo.png}
	\caption{Tiempo vs $\alpha$ con PCA + KNN}
	\label{fig:Tiempo vs Alpha con PCA + KNN}
\end{figure}

\begin{figure}[H]
	\centering	\includegraphics[width=0.8\textwidth]{img/big_alfa_pca_tiempo.png}
	\caption{Big Tiempo vs $\alpha$ con PCA + KNN}
	\label{fig:Big Tiempo vs Alpha con PCA + KNN}
\end{figure}

Tal como esperabamos vemos que a medida que el $\alpha$ aumenta (es decir, cuantas más componentes principales tengamos), el tiempo de ejecución también lo hace.

Luego, en línea con los resultados de los gráficos anteriores (Accuracy vs $\alpha$) podemos volver a afirmar que un $\alpha$ cercano a 10 sería un buen balance. En este gráfico notamos si tomaramos $\alpha$ = 30, tardaría aproximadamente el triple y no obtendríamos una mejora sustancial en el accuracy.
%%%%%%%%%%%%%%%%%%%%%%%%%%%%%%%%%%%%%%%%%%%%%%%%%%%%%%%%%%%%%%%%%%%%%


\begin{figure}[H]
	\centering	\includegraphics[width=0.8\textwidth]{img/k_knn_accu.png}
	\caption{Accuracy vs K con KNN}
	\label{fig:Accuracy vs K con KNN}
\end{figure}

\begin{figure}[H]
	\centering	\includegraphics[width=0.8\textwidth]{img/big_k_knn_accu.png}
	\caption{Big Accuracy vs K con KNN}
	\label{fig:Big Accuracy vs K con KNN}
\end{figure}

Dados los resultados, en este caso consideramos que utilizando un valor de K cercano a 10 obtenemos la mejor relación (dentro de nuestro set de tests).\newline
Por un lado evitamos el problema que ocurre cuando K es demasiado grande y por otro, tomamos una cantidad de imágenes cercanas suficiente como para minimizar el impacto de algún outsider. Aun que cabe destacar que en este caso particular K = 1 tuvo un mejor comportamiento de lo que esperabamos, consideramos que sería arriesgado tomarlo como valor confiable con otros sets de imágenes.

%%%%%%%%%%%%%%%%%%%%%%%%%%%%%%%%%%%%%%%%%%%%%%%%%%%%%%%%%%%%%%%%%%%%%

\begin{figure}[H]
	\centering	\includegraphics[width=0.8\textwidth]{img/k_knn_tiempo.png}
	\caption{Tiempo vs K con KNN}
	\label{fig:K vs Tiempo con KNN}
\end{figure}
\begin{figure}[H]
	\centering	\includegraphics[width=0.8\textwidth]{img/big_k_knn_tiempo.png}
	\caption{Big Tiempo vs K con KNN}
	\label{fig:Big K vs Tiempo con KNN}
\end{figure}

En estos tests obtuvimos resultados coherentes con lo esperado.
Lo único que podemos señalar es un tiempo mayor en PCA a lo que uno podría intuir, pero esto se debe a la aplicación de PCA en cada ejecución. En un uso real esperaríamos encontrar una reducción del tiempo de ejecución de KNN+PCA en relación al KNN.

\begin{figure}[H]
	\centering	\includegraphics[width=0.8\textwidth]{img/k_pca_tiempo.png}
	\caption{K vs Tiempo con KNN + PCA}
	\label{fig:K vs Tiempo con KNN + PCA}
\end{figure}
\begin{figure}[H]
	\centering	\includegraphics[width=0.8\textwidth]{img/big_k_pca_tiempo.png}
	\caption{Big Tiempo vs K con KNN + PCA}
	\label{fig:Big K vs Tiempo con KNN + PCA}
\end{figure}


%%%%%%%%%%%%%%%%%%%%%%%%%%%%%%%%%%%%%%%%%%%%%%%%%%%%%%%%%%%%%%%%%%%%%
\begin{figure}[H]
	\centering
	\includegraphics[width=0.8\textwidth]{img/k_pca_accu.png}
	\caption{Accuracy vs K con KNN + PCA}
	\label{fig:K vs Accuracy con KNN + PCA}
\end{figure}

En este caso vemos una estrecha relación entre cuantos vecinos cercanos tomamos y el accuracy.
Esto se debe a que al tomar más vecinos cercanos nos exponemos a un error mayor debido a que le estaríamos dando el mismo peso a todos esos K vecinos sin importar que tan cerca estén de la imagen testeada.
Llevando esto a un extremo podríamos tomar $K$ = “Tamano de matriz de entrenamiento”  cualquiera de las clases tendría el mismo peso con lo que perdería sentido este método.
Por otro lado tampoco es conveniente tener un $K$ demasiado chico. Por ejemplo, si tomaramos $K = 1$ asociaríamos la imagen a testear con la que esté a menor distancia, que debido a alguna diferencia la forma en que fue tomada la imagen puede no pertenecer a la clase de la imagen testeada.

Es llegamos a la conclusión de que utilizando un valor de K cercano a 10 obtenemos la mejor relación (dentro de nuestro set de tests).
Por un lado evitamos el problema que ocurre cuando K es demasiado grande y por otro, tomamos una cantidad de imágenes cercanas suficiente como para minimizar el impacto de algún outsider.


%%%%%%%%%%%%%%%%%%%%%%%%%%%%%%%%%%%%%%%%%%%%%%%%%%%%%%%%%%%%%%%%%%%%%
\begin{figure}[H]
	\centering	
	\includegraphics[width=0.8\textwidth]{img/acu_pre.png}
	\caption{Accuracy y precision vs K}
	\label{fig: Accuracy y precision vs K con KNN}
\end{figure}
En este último experimento estudiamos la forma en la que la cantidad de vecinos cercanos afecta a las métricas Accuracy y Precisión. Para esto utilizamos el método KNN (sin PCA) para no involucrar más variables dentro del experimento de las necesarias.

Lo que encontramos no fue muy distinto de lo esperado. Antes ya vimos la forma en la que Accuracy variaba en función de la cantidad de vecinos cercanos (Figura 5). Pero como explicamos en cuanto al funcionamiento y elección de un K apropiado para el KNN, un K = 250 por ejemplo no es una buena elección, así que en este caso el accuracy resulta ser una métrica un tanto engañosa.

Para tener un sistema preciso -valga la redundancia- necesitamos un valos de precision relativamente alto. Entonces en función de lo que nos indica el gráfico nuevamente un valor aproximado de K = 10 nos parece una buena opción.
% \input{discucion.tex}
% \input{conclusiones.tex}
% \subsubsection*{Apéndices}
\textbf{Estructuras de Datos}

Para manejar las imágenes implementamos la clase Imagen, en la cual almacenamos los parámetros de cada imagen (la altura y ancho originales de la imagen, el nombre del archivo, id de la clase y los bytes de la imagen propiamente dicha en formato vector de unsigned char).
Para la lectura de los archivos de imágenes utilizamos las librerías provistas por la cátedra.
Las imágenes a medida que se leen se almacenan en la estructura que denominamos baseDeDatos, la cual consiste de un vector de Imagen.
Para la implementación de las funciones necesarias para PCA trabajamos con dos estructuras principales: doubleVector que como lo indica su nombre consiste de un vector de doubles donde almacenamos cada imagen vectorizada, y la otra estructura es dobleMatrix que usamos para almacenar todas las imágenes vectorizadas (una por fila), con el objetivo de poder operar matricialmente con ellas dentro de las funciones de una forma práctica.


% \input{referencias.tex}
 

\end{document}
