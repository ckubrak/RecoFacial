\subsubsection*{Introducción}
\par El problema que se nos platea es el de realizar un reconocimiento facial mediante un aprendizaje supervisado. Partimos de una base de datos, a la cual llamaremos \textit{base de entrenamiento},
que consiste en un conjunto de $N$ personas de las cuales contamos con $M$ fotos diferentes de sus caras. Al recibir una nueva imagen, buscamos identificar
a qu\'e persona le correspone. 
\par Para reconocer la nueva cara experimentaremos con dos m\'etodos: el primero usando los $k$ vecinos m\'as cercanos (\textit{kNN}) y 
el segundo utilizando el an\'alisis de componentes principales (\textit{PCA}) como forma de preprocesar la \textit{base de entrenamiento} para reducir el 
tama\~{n}o de la imagen y luego correr \textit{kNN}. 
\par Por \'ultimo, para evaluar los m\'etodos y la correcta elecci\'on de par\'ametros utilizaremos una t\'ecnica de \textit{cross validation} llamado
\textit{K-fold}.

\subsubsection*{K vecinos m\'as cercanos}
\par A partir de la \textit{base de entrenamiento} buscamos identificar a qu\'e sujeto pertenece una nueva cara sin identificar.
Para este algoritmo considera a cada imagen de la \textit{base de entrenamiento} como un vector de dimensi\'on $n$, donde $n = altura*ancho$ asumiendo que
todas las im\'agenes tienen el mismo tama\~no, para el cual se conoce a cu\'al es el ID de la imagen (i.e.: la persona) para luego mediante el c\'alculo
de la nomrma de la diferencia, obtener los $k$ elementos m\'as cercanos.\\
Esta forma de encarar el problema resuta poco pr\'actica cuando la dimensi\'on de la im\'agen es grande, es por esto que en ciertos casos preprocesamos
la base de datos con el m\'etodo \textit{PCA}.

\subsubsection*{An\'alisis de componentes principales}
\par Para este algoritmo de preprocesamiento, calculamos $\mu = (\sum_{i=1}^{n})/n$ el promedio de todas la im\'agenes. Luego definimos $X \in {\rm I\!R}^n$ 
que en i-\'esima fila tiene al vector $(x_i -  \mu)/\sqrt{n -1}$. Para esta matriz $X$, calculamos la matriz de covarianza como $M = X^tX$.
Sea $v_j$ el autovector de $M$ asociado al j-\'esimo autovector (teniendo los autovalores ordenados seg\'un su valor absoluto), definimos para cada
$i = 1,..,n$ la \textit{transformaci\'on caracter\'istica} de $x_i$ como el vector $tc(x_i) = (v_1x_i,v_2x_2,..,v_\alpha x_i)^t \in {\rm I\!R}^\alpha$,
donde $\alpha$ es un par\'ametro que fuimos variando para la experimentaci\'on.

\subsubsection*{Cross validation} 
\par Para poder evaluar qu\'e tan bien funciona nuestro algoritmo es necesario realizar una validaci\'on de los resultados. Para esta tarea ultilizamos el
m\'etodo de \textit{Cross validation} llamado \textit{k-fold}. Subdividimos nuestra \textit{base de entrenamiento} en $k$ subconjuntos con igual cantidad
de elementos. Uno de estos lo utilizaremos como datos de prueba mientras que los $k-1$ restantes ser\'an la \textit{base de entrenamiento}. A partir de esto
podemos analizar los resultados ya que contamos la informaci\'on sobre a qu\'e sujeto pertenece cada im commagen de nuestra base de pruebas. Por \'ultimo resta repetir
el procedimiento para que cada uno de los subconjuntos sea considerado como base de pruebas.

\subsubsection*{Criterios de evaluaci\'on}
Resulta de vital importancia tener en claro cu\'ales son los criterios a tener en cuenta para ver si nuestra clasificaci\'on es "buena" o no.
Para poder evaluarla, debemos mirar los \textit{true positives} y \textit{true negatives} (los casos en los cuales el encasillado funcion\'o correctamente)
como tambi\'en los \textit{false positives} y \textit{false negatives} (casos en los cuales la clasificaci\'on no se comport\'o como se esperaba).\\
Siendo el caso a analizar no binario (existen m\'as de dos categor\'ias), para cada $i = 1,..,n$  definimos $tp_i$ como
las muestras que realmente pertenec\'ian a la clase $i$ y fueron exitosamente identificadas como tales
y $fn_i$ son aquellas muestras que fueron identificadas como pertenecientes a la clase i cuando realmente no lo eran (an\'alogamente definimos $tn_i$ y $fp_i$).
\par A partir de esto definimos cuatro m\'etricas, a saber:
$precision_i= \frac{tp_i}{tp_i + fp_i}$ la cual indica qu\'e porcentaje de los datos recuperados cu\'ales son relevantes. 
El $recall_i= \frac{tp_i}{tp_i + fn_i}$, que indica qu\'e porcentaje de los datos relevantes fueron recuperados.
El $accuracy_i = \frac{tp_i+tn_i}{tp_i + tn_i + fp_i + fn_i}$ indica la cantidad de aciertos sobre el total de la base de datos.
Por \'ultimo la media arm\'onica $F_{1_i} = 2\frac{precision_i * recall_i}{precision_i + recall_i}$ permite establecer un compromiso entre en 
$recall_i$ y el $accuracy_i$.