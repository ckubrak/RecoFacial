\subsubsection*{Introducción}
\par El problema que se nos platea es el de realizar un reconocimiento facial. Partimos de una base de datos, a la cual llamaremos \textit{base de entrenamiento},
que consiste en un conjunto de $N$ personas de las cuales contamos con $M$ fotos diferentes de sus caras. Al recibir una nueva imagen, buscamos identificar
a qu\'e persona le correspone. 
\par Para reconocer la nueva cara experimentaremos con dos m\'etodos: el primero usando los $k$ vecinos m\'as cercanos (\textit{kNN}) y 
el segundo utilizando el an\'alisis de componentes principales (\textit{PCA}) como forma de preprocesar la \textit{base de entrenamiento} para reducir el 
tama\~{n}o de la imagen y luego correr \textit{kNN}. 
\par Por \'ultimo, para evaluar los m\'etodos y la correcta elecci\'on de par\'ametros utilizaremos una t\'ecnica de \textit{cross validation} llamado
\textit{K-fold}.

\subsubsection*{K vecinos m\'as cercanos}
A partir de la \textit{base de entrenamiento} buscamos identificar a qu\'e sujeto pertenece una nueva cara sin identificar.
Para este algoritmo considera a cada imagen de la \textit{base de entrenamiento} como un vector de dimensi\'on $n$, donde $n = altura*ancho$ asumiendo que
todas las im\'agenes tienen el mismo tama\~no, para el cual se conoce a cu\'al es el ID de la imagen (i.e.: la persona) para luego mediante el c\'alculo
de la nomrma de la diferencia, obtener los $k$ elementos m\'as cercanos.\\
Esta forma de encarar el problema resuta poco pr\'actica cuando la dimensi\'on de la im\'agen es grande, es por esto que en ciertos casos preprocesamos
la base de datos con el m\'etodo \textit{PCA}.


\subsubsection*{An\'alisis de componentes principales}
Para este algoritmo de preprocesamiento, calculamos $\mu = (\sum_{i=1}^{n})/n$ el promedio de todas la im\'agenes